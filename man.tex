\input texinfo
@settitle Maze Generator Manual

@copying
This manual is for Maze project.

Copyright @copyright{} 2022 bperegri irobin
@end copying

@titlepage
@title Maze Generator Manual
@author by bperegri, irobin
@page
@vskip 0pt plus 1filll
@insertcopying
@end titlepage

@contents

@node Top

@menu
* General principles : GeneralPrinciples. What is this all about?
* Load, Save and Render a Maze : Implementation. Simple functions for drawing a maze.
* Perfect Maze Generation : Generation. How to generate a new random maze.
* Solving the Maze : Pathfinding. Searching for the shortest path between two points in the maze.
* Caves :: Save, Load, Generate.
@end menu

@node GeneralPrinciples, Implementation, Top, Top
@chapter General principles
  This programm allows creating mazes and caves and browsing those saved in files in a predefined formats.

  Left part of the programm's interface contains a field meant for graphical representation of mazes or caves.

  Mazes are rectangular, with thin walls, enclosed. Caves consist of square wall tiles(black) and free tiles(white).

  Right part of the programm's interface contains two tabs, in which all controll elements related to mazes and to caves are separated.

@node Implementation, Generation, GeneralPrinciples, Top
@chapter Load, Save and Render a Maze
  You can load a maze from a file with "Load" button. If the file doesn't contain a correct maze nothing will be loaded.

  You can save a maze in a file with "Save" button.

  Current maze will be displayed in a graphics field.

@node Generation, Pathfinding, Implementation, Top
@chapter Perfect Maze Generation
  You can generate a random perfect maze by clicking "Generate" button in the Maze tab.

  Perfect - means the generated maze will have a path between any two tiles and no cycles.

  Maze tab contains parameters for maze generation: width and heigth. Limits are 1 - 50 for each. You can set them by sliders or using keyboard.

@node Pathfinding, Caves, Generation, Top
@chapter Solving the Maze
  The programm can find a shortest path between any two tiles if exists.

  Set the start point by clicking a tile with left mouse button. It will appear as a green ellipse.

  Once the start point is set, click another tile to set or change finish. If there is a valid path between these points it will be represented by green polyline. If not - start and finish will be shown as red ellipses. Note, that generated maze is perfect, so there will always be a path. On the other hand, mazes loaded from files could be non-perfect.

  To reset start and finish points click the maze with right mouse button.

@node Caves, Top, Pathfinding, Top
@chapter Caves
  Caves can be saved, loaded and generated just as Mazes, but using tab Cave.

  Use sliders or fields to change cave's sizes.

  First step of caves generation is filling it randomly with walls. "initial probability" parameter sets the probability in percents (0 - 100) for a tile to be a wall at this step.

  Then go Iterations: some walls will turn to free spaces, some free spaces will turn to walls according to following rules.

  If free space has more nearby walls than "birth limit" it will turn to a wall. If a wall has less wall tiles nearby than "death limit" it will turn to a free space. Nearby means tiles next to it horizontally, vertically or diagonally.

  You can start/stop auto-iterations by "Auto"/"Stop" button. Or do each step manually, by clicking "Step"

  Auto-iteration delay can be set by a slider and a field in a range 0 - 1000 ms.

@bye